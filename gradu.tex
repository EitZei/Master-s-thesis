% Tarvitaanko twoside?
% Oleellista: utathesis sisältää tyylit
% t1enc sisältää latin-1 koodauksen
% babel sisältää suomennetut nimet
% named on suositeltu bibliografia tyyppi

\documentclass[twoside]{utathesis}
\usepackage[T1]{fontenc}
\usepackage[utf8]{inputenc}
\usepackage[finnish]{babel}
\usepackage{named}

% Suositus on, että kuvat erotetaan viivoilla
\usepackage{utafloat}
\usepackage{url}

\begin{document}

% Erotetaan kuvat viivoilla
\floatstyle{ruled}
\restylefloat{figure}

\prelimpages

\Title{Gradu}
\Author{Antti Järvinen}
\Supervisor{Timo Niemi}
\titlepage

\setcounter{page}{-1}
\abstract{%
Abstract goes here.%
}

\tableofcontents
%\listoffigures % not necessary at all
%\listoftables  % I have no tables

\textpages
 
\chapter{Introduction}
Testing citing \cite{brewer}

\chapter{Scaling}
\section{Vertical scaling}
i.e. scaling up
\section{Horizontal scaling}
i.e. scaling out

\chapter{CAP}

\chapter{Scaling out RDBMS}

\chapter{Alternatives}
\section{NoSQL}
\section{Datamodels}
\section{Cassandra, BigTable etc.}
\section{Applications}
\section{Pros and Cons}

\input{discussion}

%\nocite{*}   % include everything in gradu-esim.bib file
\bibliographystyle{named}
\bibliography{bib}

% Tämä on utathesis paketin bugi: Tarvitaan end-of-document viite, että
% sivujen määrä saadaan laskettua oikein
\label{end-of-document}
\end{document}